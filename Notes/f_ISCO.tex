\documentclass{article}
\usepackage[letterpaper, portrait, margin=1in]{geometry}
\usepackage{amsmath, amssymb, bm, wasysym}

\newcommand{\isco}{\text{ISCO}}
\newcommand{\tot}{\text{tot}}

\begin{document}
\flushleft

\begin{center}
\Large{GW Frequency at ISCO Using Two-Body Decomposition}
\end{center}
\vspace{1cm}
The radius of the innermost stable circular orbit (ISCO) for a non-rotating black hole of mass $M_{\text{BH}}$ with a test particle orbiting it is given by
\begin{equation}\label{isco}
r_{\isco}=\frac{6\,G\,M_{\text{BH}}}{c^{2}}.
\end{equation}
Kepler's third law for two bodies with masses $M_{1}$ and $M_{2}$ is
\begin{equation}
T^{2}=\frac{4\,\pi^{2}}{G\left(M_{1}+M_{2}\right)}a^{3},
\end{equation}
where $T$ is the orbital period and $a$ is the semi-major axis. If we first transform coordinates using the two-body decomposition and evaluate this at the ISCO we get
\begin{equation}\label{Kepler}
T^{2}=\frac{4\,\pi^{2}}{G\left(M_{\tot}+\mu\right)}r_{\isco}^{3},
\end{equation}
where
\begin{equation}
M_{\tot}=M_{1}+M_{2}\hspace{3em}\text{and}\hspace{3em}\mu=\frac{M_{1}\,M_{2}}{M_{1}+M_{2}}.
\end{equation}
Substituting \eqref{isco} in \eqref{Kepler} we get
\begin{equation}
T^{2}=\frac{4\,\pi^{2}}{G\left(M_{\tot}+\mu\right)}\left(\frac{6\,G\,M_{\tot}}{c^{2}}\right)^{3}=\frac{6^{3}\,4\,\pi^{2}\,G^{2}}{c^{6}}\frac{M_{\tot}^{3}}{M_{\tot}+\mu}.
\end{equation}
Now, $T$ is the orbital period, which is twice the period of the emitted gravitational waves at ISCO,
\begin{equation}
T=2\,T_{\isco}\implies T=\frac{2}{f_{\isco}}.
\end{equation}
So, plugging that into the above expression we get
\begin{equation}
\frac{4}{f_{\isco}^{2}}=\frac{6^{3}\,4\,\pi^{2}\,G^{2}}{c^{6}}\frac{M_{\tot}^{3}}{M_{\tot}+\mu}\implies f_{\isco}^{2}=\frac{c^{6}}{6^{3}\,\pi^{2}\,G^{2}}\frac{M_{\tot}+\mu}{M_{\tot}^{3}}.
\end{equation}
Taking the square root of both sides yields
\begin{equation}
f_{\isco}=\frac{c^{3}}{6^{3/2}\,\pi\,G}\left(\frac{M_{\tot}+\mu}{M_{\tot}^{3}}\right)^{1/2}.
\end{equation}
Now, this quantity was evaluated at the location of the orbiting bodies (assumed to be at redshift $z$). The frequency we actually observe will be redshifted and will appear longer by a factor of $\left(1+z\right)$. So our final result is
\begin{equation}
\boxed{f_{\isco,\text{obs}}=\frac{c^{3}}{6^{3/2}\,\pi\,G}\left(\frac{M_{\tot}+\mu}{M_{\tot}^{3}}\right)^{1/2}\times\frac{1}{1+z}.}
\end{equation}
We note that in the limit of $M_{\tot}\gg\mu$, this result agrees with Eq. (12) in Sesana, et al. (2005).
\end{document}